\pretolerance=1400
\magnification=\magstep1
%\baselineskip=0.77cm
\hsize=4.8in
%\hoffset=0.4in
%\voffset=0.2in
%\font\mb=cmsy10
%\font\cp=cmr7
%\font\bcp=cmbx7
%\font\icp=cmmi7
\def\und{\underbar}
\def\sp{\scriptstyle}
\def\spp{\scriptscriptstyle}

\quad
\centerline {\bf Spatial Interpolation }
\bigskip
\centerline {\sl Helena Mitasova, NCSU, lecture notes for 
MEA592 Geospatial Analysis and modeling} 
\bigskip
\noindent
The problem is formulated as follows:

Given the $m$ values of a studied phenomenon
 $z_j,\; j=1, \dots ,m$ measured
at discrete points ${\bf r}_j=(x_j,y_j),
\; j=1,\dots ,m$ within a certain region of 2-dimensional space,
 find a function $F({\bf r})$ which fulfils the following condition:
$$
F({\bf r}_j)=z_j,\quad j=1,\ldots ,m \eqno(1)
$$
The problem does not have a unique solution so additional conditions
are used. 

% add here approximation condition

However, most interpolation
methods can be expressed as a sum of two components
$$
F({\bf r})=T({\bf r})+\sum_{j=1}^m \lambda_j R({\bf r},{\bf r}_j)\, 
 \eqno(2)
$$
where $T({\bf r})$ is a `trend' function, $\lambda_j$ are weights 
(coefficients) and $R({\bf r},{\bf r}_j)$ is a function of distance between 
an unsampled and a measured point. The explicit form of $R(.)$
depends on the choice of the additional conditions
(locality, ad hoc, geostatistical -based on variogram, physics - minimization
of energy, etc.)

\bigskip
\noindent
\centerline {\bf Voronoi polygons}
\smallskip

First, Voronoi polygons are created from the measured data (each
measured point will be at the center of a Voronoi polygon) .
Then, $z$-value at a point ${\bf r}$ will be the same as the $z({\bf r_j})$-value at 
the center of the Voronoi polygon $V_j$ within which the  
point ${\bf r}$ is located.
That means for our general equation, that $T({\bf r})=0$ and
$\lambda =1, {\bf r} \in V_j$ and $\lambda =0, {\bf r} \notin V_j$
leading to 

$$
z({\bf r}) = \lambda z({\bf r_j})
 \eqno(3)
$$
\noindent
The resulting surface is not continuous and it includes only 
the measured values.

\bigskip
\noindent
\centerline {\bf TIN-based linear interpolation}
\smallskip

%linear, Akima, non-linear (natural neighbor, Bezier, ...)
First, TIN is created from the measured data using Delaunay triangulation
(each measured point will be a triangle vertex).
Then, $z$-value at an unsampled point ${\bf r}$ 
(we denote it as point $K$ within a triangle $ABC$)
will be computed as a function 
of the $z$-values at the vertices of the triangle within which 
the point ${\bf r}$ is located. The function is a weighted average,
where the weights are function of areas of triangles $ABK, ACK, BCK$.
That means for our general equation, that $T({\bf r})=0$ and
$\lambda_k =p_k$ where $p_k, k=1,2,3$ are areas of sub-triangles
$ABK, ACK, BCK$ leading to 

$$
z({\bf r}) = 
{p_{BCK} z({\bf r_A}) + p_{ACK} z({\bf r_B}) + p_{ABK} z({\bf r_C}) \over {p_{ABC}}}
$$
or
$$
z({\bf r}) ={{\sum_{k=1}^3 p_k z({\bf r}_k)} \over {p_{ABC}}
 \eqno(4)
$$
\noindent
It can be also interpreted as point on a plane defined by the vertices 
of the triangle $ABC$.
The resulting surface is continuous, but its first derivatives are not
($C^0$). 

There are many non-linear interpolation methods for TINs that lead to
$C^1$ or $C^2$ continuous surface. 

\bigskip
\centerline {\bf Inverse distance weighted interpolation (IDW)}
\smallskip

The method
is based on an assumption that the value at an unsampled point can be
approximated as a weighted average of values at measured points within a
certain cut-off distance, or from a given number $m$ of the closest
points (typically 10 to 30).  Weights are usually inversely
proportional to a power of distance leading to an estimator:

\smallskip
$$ 
z({\bf r})=\sum_{i=1}^m w_i z({\bf r}_i)=
{ \sum_{i=1}^m
{z({\bf r}_i)/|{\bf r}- {\bf r}_i|^p}
\over
{\sum_{j=1}^m 1/|{\bf r} - {\bf r}_j|^p}
}
\eqno(5)
$$
\smallskip \noindent
where $p$ is a parameter (typically $p=2$, lower $p$ gives 
smoother surface  - similar to lower tension), and
$|{\bf r}- {\bf r}_i|^2 = (x-x_i)^2+(y-y_i)^2$ is the distance
between the unsampled location ${\bf r}$ and a given point ${\bf r}_i$.

Smoothing can be introduced by adding a parameter $\beta$
to the weight term $|{\bf r} - {\bf r}_j|^2+\beta$, 
leading to approximation function.

Note: GRASS modules use $p=2$ and $m=12$ as default values.

\bigskip
\noindent
\centerline {\bf Geostatistical approach: Kriging}
\smallskip
Kriging is based on a concept of random functions: the surface
or volume is assumed to be one realization of a random function
with a certain spatial covariance.
Using the given data $z({\bf r}_i)$ and an assumption of stationarity
one can estimate a semivariogram $\gamma({\bf h})$ (Getis 1997 P3.2), defined as
$$
\gamma({\bf h})={1\over 2}
Var \left[ \{ z({\bf r}+{\bf h})-z({\bf r}) \} \right]
\approx 
 {1 \over {2N_h}} \sum_{(ij)}^{N_h} [z({\bf r}_i)-z({\bf r}_j)]^2
\eqno(6)
$$
 which is related to the spatial covariance $C({\bf h})$ as
$$
\gamma ({\bf h})= C(0) - C({\bf h}) \eqno (7)
$$
where $C(0)$ is the semivariogram value at infinity (sill).
 The summation in Eq. (6) runs over the number $N_h$ of pairs of points
 which are separated by the vector {\bf h} within a small tolerance
  $\Delta {\bf h}$ (size of a histogram bin).
For isotropic data, the semivariogram can be simplified into a radial function
dependent on $|{\bf h}|$. The kriging literature provides
 a choice of functions  which can be used as theoretical semivariograms
 (spherical, exponential, Gaussian, Bessel, etc. ) 
 The parameters of these functions
 are then optimized for the best fit of the experimental semivariogram.

The interpolated surface is then constructed using
 statistical conditions of unbiasedness and minimum variance.
 In its dual form (Matheron 1971, Hutchinson and Gessler 1993)
the universal kriging interpolation
function can be written as
$$
F({\bf r})=T({\bf r})+\sum_{j=1}^N \lambda_j C({\bf r}-{\bf r}_j)
 \eqno(5)
$$
where $T({\bf r})$ represents its non-random component (drift) expressed as a
 linear combination of low-order monomials. The
 monomial and $\{\lambda_j\}$ coefficients
are found by solving a system of linear
 equations (Hutchinson and Gessler 1993).

\bigskip
\noindent
\centerline {\bf Radial basis functions and splines}
\smallskip

Variational approach to interpolation and approximation
 is based on the assumption that the
interpolation function should pass through (or closely to)
the data points and, at the same time, should be as smooth as possible.
These two requirements are combined into a single
condition of minimizing the sum of the deviations from the measured points
and the smoothness seminorm of the spline function:
$$
\sum_{j=1}^N |z_j-F({\bf r}_j)|^2w_j + w_0I(F)={\sl minimum}  
\eqno(6)
$$
where $w_j,w_0$ are positive weights
 and $I(F)$ denotes the smoothness
seminorm (Table 1).
The solution of Eq. (6)
can be expressed as a sum of two components (Talmi and Gilat 1977, Wahba 1990)
$$
F({\bf r})=T({\bf r})+\sum_{j=1}^N \lambda_j R({\bf r},{\bf r}_j)
 \eqno(7)
$$
where $T({\bf r})$ is  a 'trend' function and
 $R({\bf r},{\bf r}_j)$
is a basis function which has a form dependent
on the choice of $I(F)$.
Bivariate smoothness seminorm with squares of
second derivatives (Table 1) leads to
 a {\bf thin plate spline (TPS)}
 function, when first derivatives are added we get
{\bf thin plate spline with tension}, when all derivatives 
are used with decreasing weight we get
{\bf completely regularized spline} or 
{\bf regularized spline with tension}. 

Thin plate spline with tension is then
$$
F({\bf r})=a_0 + \sum_{j=1}^N \lambda_j 
\left[ {\rm K_0} (\varphi r/2) + \ln(\varphi r/2) + C \right] 
$$
where $K_0(.)$ is the Bessel? function and $\varphi$ is tension parameter,
$C$ is constant.

Regularized spline with tension is given as
$$
F({\bf r})=a_0 - \sum_{j=1}^N \lambda_j 
\left[ {\rm E_1} (\varrho) + \ln\varrho + C_E \right]
$$
where $C_E=0.577215...$ is the Euler constant, ${\rm E}_1(\varrho)$ is
the exponential integral function, while the trend function is a
constant and $r=|{\bf r} - {\bf r}_j|$, $\varrho = (\varphi r/2)^2$
and $\varphi$ is a tension parameter.

The coefficients $a_1, \{\lambda _j\}$ are obtained by solving the following
system of linear equations:

$$
 a_1+\sum_{j=1}^N \lambda_j \left[R({\bf r}^{[i]},{\bf r}^{[j]})
+\delta_{ji}w_0/w_j \right] =z^{[i]}\, ,\quad i=1,\ldots,N
$$

$$
\sum_{j=1}^N \lambda _j =0\, .
$$
where $w_0/w_j$ are positive smoothing weights.


While not obtained by variational approach, similar in
formulation and performance are
{\bf multiquadrics} (Hardy 1990, Fogel 1996, Kansa and Carlson 1992,
 Franke 1982a, Foley 1987, Nielson 1993)  with $R_d(r)=(r^2+b)^{1\over 2}$
or $R_d(r)=(r^2+b)^{-{1\over 2}}$,
offering high accuracy, differentiability, $d$-dimensional formulation,
and, with segmentation,
 applicability to large data sets. Originally ad hoc proposed
 multiquadrics were later put on a more solid theoretical ground.
Good performance of multiquadrics, especially in 3D, is not surprising,
considering that for $d=3$ in the limit of $b \to 0$  the basis
functions $(r^2+b)^{1\over 2}$ and $(r^2+b)^{-{1\over 2}}$
are solutions of biharmonic and harmonic equations, respectively
(Hardy 1990).



\bye

