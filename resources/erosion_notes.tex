\pretolerance=1400
\magnification=\magstep0
\baselineskip=0.68cm
\hsize=5.4in
%\hoffset=0.4in
%\voffset=0.2in
%\font\mb=cmsy10
%\font\cp=cmr7
%\font\bcp=cmbx7
%\font\icp=cmmi7
\def\und{\underbar}
\def\sp{\scriptstyle}
\def\spp{\scriptscriptstyle}
\def\tabrule{\noalign{\vskip 5truept \hrule\vskip 5truept} }
\def\tabrul2{\noalign{\vskip 5truept \hrule\vskip 2truept \hrule\vskip 5truept} }
\def\fbb {~\bar{\kern-0.325em \bar f}}
\def\bnabla{\mb \nabla}

\centerline {\bf Erosion modeling}
Universal Soil Loss Equation (USLE) general equation:
$$
E = R K L S C P
$$

\noindent
where 
$E$ is average annual soil loss in $ton/(acre.year)=0.2242kg/(m^2.year)=2.242ton/(ha.year)$, 

\noindent
$R$ is rainfall factor in (hundreds of ft-tonf.in)/(acre.hr.year) = 17.02(MJ.mm)/(ha.hr.year),

\noindent
$K$ is soil erodibility factor in (ton acre.hr)/(hundreds of acre ft-tonf.in) = 

\noindent
= 0.1317(ton.ha.hr)/(ha.MJ.mm),

\noindent
$LS$ is a dimensionless topographic (length-slope) factor,

\noindent
$C$ is a dimensionless land cover factor, and

\noindent
$P$ is a dimensionless prevention measures factor.

\noindent
The modified  3D factor, representing topographic potential
for erosion {\bf at a point on the hillslope}, is
a function of the upslope area per unit width $U$ and the slope angle:
$$
LS=(m+1) \left(U/22.1\right)^m \left({\sin{\beta}}/{0.09}\right)^n
$$
\noindent
where 

\noindent
$U$ is the upslope area per unit width (measure of water flow) in meters ($m^2/m$), 

\noindent
$\beta$ is the slope angle in degree, 

\noindent
$22.1$ is the length of the standard USLE plot in meters, 

\noindent
$0.09 = 9\% = 5.15^{\circ}$ is the slope of the standard USLE plot. 

\noindent
The values of exponents range for $m=0.2-0.6$ and
$n=1.0-1.3$, where the lower values are used for prevailing sheet flow and
higher values for prevailing rill flow.  

\smallskip
{\bf Unit Stream Power Based Erosion/Deposition model (USPED)}

\noindent
 USPED estimates a sediment transport limited case of erosion/deposition 
using the concept proposed by Moore and Burch (1986). 
It combines the USLE/RUSLE parameters
and upslope contributing area per unit width $U$ to estimate the
sediment flow $T$ at sediment transport capacity:
$$
 T \approx R K C P U^m (sin \beta)^n.
$$
\noindent
$R, K, C, P, U, \beta$ are the same as in USLE,
and $U, \beta$ are not normalized, because 

\noindent
$T$ is an estimate of sediment flow $[kg/(m.s)]$ or
$[ton.m/(ha.year)] = [ton/(10000m.year)]$.
(rather than soil loss E $[kg/(m^2.s)]$).

\noindent
The net erosion/deposition $D$ is then computed as a divergence of sediment
flow (change in a 2d field representing sediment flow in the direction of
elevation surface gradient):
$$
D=\nabla\cdot (T{\bf s_0}) = {\partial (T \cos \alpha) \over {\partial x}} +
{\partial (T \sin \alpha) \over {\partial y}},
$$
\noindent
where 

\noindent
$\alpha$ in degrees is the aspect of the terrain surface (direction of flow).

\noindent
We get $D$ in $[kg/(m^2s)]$ by dividing $T[kg/(ms)] / dx[m]$ 
or $T[ton/10000m.year]/dx[m]=D[ton/(10000m^2. year)]=D[(ton/ha.year)]$

The exponents $m,n$ control the relative influence of water and slope
terms and reflect the impact of different types of flow. The typical range of
values is $m=1.0-1.6, n=1.0-1.3$, with the higher values reflecting the pattern
for prevailing rill erosion with more turbulent flow when erosion sharply
increases with the amount of water.  Lower exponent values close to $m=n=1$
better reflect the pattern of compounded, long term impact of both rill and
sheet erosion and averaging over a long term sequence of large and small
events.

We can transform the rate of erosion or deposition to 
elevation change at grid point:
$$
dz=D/\varrho
$$
\noindent
where
$\varrho$ is soil density in $[kg/m^3]$=$[0.001 ton / m^3]$ so we divide 
erosion/deposition rate $D$ in $[ton/(ha.year)]$ by 10-times density 
in $[kg/m^3]$

\noindent
$dz[m/year]=D/(10.\varrho)$ is elevation change per year at grid point.

\noindent
Mass of sediment eroded or deposited at a cell 

$$D_{cell}[ton/year]=D[ton/(10000m^2s)]*r^2[m^2]$$

\noindent
where $r$ is raster resolution in meters.

\bye

